%----------------------------------------------------------------------------
% Course Syllabus for Chemistry 351 (c) 2015 by Dale J. Brugh

% Course Syllabus for Chemistry 351 is released under a
% Creative Commons Attribution-ShareAlike 4.0 International License.

% See http://creativecommons.org/licenses/by-sa/4.0/ for
% a description of your rights under this license. 
%----------------------------------------------------------------------------
\documentclass[letterpaper,oneside,onecolumn,11pt,article]{memoir}
%
% --- LOAD PACKAGES ---
%
\usepackage[T1]{fontenc}            % use T1 font encoding
\usepackage{textcomp}
\usepackage{courier}                % set courier as typewriter font
\usepackage{times}                  % set times as text font
\usepackage[scaled=0.92]{helvet}    % set Helvetica as the sans-serif font
\usepackage{mtpro2}
\usepackage{setspace}
\usepackage{amsmath}
\usepackage{graphicx,color}
\usepackage{wallpaper}
\usepackage{textcomp}
\usepackage{relsize,fancyvrb}
\usepackage{verbatim}
\usepackage{caption}
\usepackage{paralist}
\usepackage{boxedminipage}
\usepackage[bookmarks=true]{hyperref}
%
% --- HYPER SETUP ---
%
\hypersetup{
    unicode=false,          % non-Latin characters in Acrobat’s bookmarks
    pdftoolbar=true,        % show Acrobat’s toolbar?
    pdfmenubar=true,        % show Acrobat’s menu?
    pdffitwindow=true,      % page fit to window when opened
    pdftitle={Syllabus: Chemistry 351 / Spring 2015}, 
    pdfauthor={Dale J. Brugh},     % author
    pdfsubject={Physical Chemistry},   % subject of the document
    pdfnewwindow=true,      % links in new window
    pdfkeywords={classes, ch351s15}, % list of keywords
    colorlinks=true,       % false: boxed links; true: colored links
    linkcolor=black,          % color of internal links
    citecolor=green,        % color of links to bibliography
    filecolor=magenta,      % color of file links
    urlcolor=black           % color of external links
}
\definecolor{nicered}{rgb}{.647,.129,.149}
\definecolor{mutedgrey}{rgb}{0.4,0.4,0.4}
\definecolor{shadecolor}{cmyk}{0,0,0.25,0.07}
\definecolor{MyDarkBlue}{rgb}{0,0.08,0.45}
\definecolor{MarginRed}{rgb}{0.8,0.0,0.0}
\definecolor{MarginBlue}{rgb}{0.2,0.0,1.0}
\definecolor{MarginGrey}{rgb}{0.4,0.4,0.4}
%\renewcommand{\chapnumfont}{\bfseries\Huge\sffamily}
%\renewcommand{\chaptitlefont}{\bfseries\Large\sffamily}
\setsecheadstyle{\bfseries\Large\sffamily\raggedright}
\setsubsecheadstyle{\bfseries\large\sffamily\raggedright}
\setsubsubsecheadstyle{\bfseries\normalsize\sffamily\raggedright}
\renewcommand \thesection{\bfseries\arabic{section}}
\makeatletter 
\newcommand\addRevisionData{%
\begin{picture}(0,0)% 
    \put(-110,-5){%
        \tiny% 
        {%
        {Published \today \enspace \copyright~Dale J. Brugh 
        }
}% 
}%
\end{picture}%
}
\flushbottom
\setstocksize{11in}{8.5in}
%\setlength{\parskip}{5pt}
\settrims{0pt}{0pt}
%\settrimmedsize{11in}{210mm}{*}
%\setlength{\trimtop}{0pt}
%\setlength{\trimedge}{\stockwidth}
%\addtolength{\trimedge}{-\paperwidth}
\settypeblocksize{8.5in}{5.0in}{*}
\setulmargins{1.25in}{*}{*}
\setlrmargins{1.25in}{*}{*}
\setmarginnotes{5mm}{4.0cm}{\onelineskip}
\setheadfoot{\onelineskip}{4\onelineskip}
\setheaderspaces{*}{\onelineskip}{*}
\checkandfixthelayout
%\setlength \fboxsep{0.1in}
\setlength \headwidth{\textwidth+\marginparwidth+\marginparsep}
%
\makepagestyle{courseinformation}
\makerunningwidth{courseinformation}{\headwidth}
\makeheadrule{courseinformation}{\headwidth}{\normalrulethickness}
\makeheadposition{courseinformation}{flushright}{flushleft}{flushleft}{flushleft}
\makeoddhead{courseinformation}%
    {\sffamily Course Syllabus: Chemistry 351 / Spring 2015}{}{\sffamily\thepage}

    \makeevenfoot{courseinformation}{}{}{}
    \makeoddfoot{courseinformation}{}{}{}
\makepagestyle{courseinformationtitle}
\makerunningwidth{courseinformationtitle}{\headwidth}
\makeheadposition{courseinformationtitle}{flushright}{flushleft}{flushleft}{flushleft}
    \makeevenfoot{courseinformationtitle}{}{}{}
    \makeoddfoot{courseinformationtitle}{}{}{}
\pagestyle{courseinformation}
\captionsetup{labelsep=colon,aboveskip=0.25cm,justification=RaggedRight,singlelinecheck=false,labelfont={bf,sf}}
%
% --- MARGIN FIGURE COMMAND ---
%
\newcommand{\marginfigures}[4]{
\marginpar{\centering
\includegraphics[width=#1]{#2}
\captionsetup{labelsep=newline,aboveskip=-0.5cm,justification=RaggedRight,singlelinecheck=false,labelfont={bf,sf}}
\captionsetup[figure]{position=bottom}
\captionof{figure}{#3}
\label{#4}}%
}%
%
% --- MARGIN NOTE COMMAND ---
%
\newcommand{\marginnote}[2]
{%
\marginpar{\raggedright\vspace{#1}\begin{Spacing}{0.65}\sffamily{{\tiny$\blacktriangleright$~\scriptsize#2}}\end{Spacing}} %
}
%
% --- SET UP TITLE ---
%
\setlength{\droptitle}{0.0in}
\backmatter
\pretitle{\noindent\huge\sffamily Course Syllabus \LARGE\par\noindent} 
\posttitle{\par\vskip 2.0em}
\preauthor{}
\postauthor{\par}
\predate{}
\postdate{\noindent\rule{\linewidth}{0.3pt}}
\title{Chemistry 351 / Spring 2015}
\date{}
\author{}
%
% --- BEGIN DOCUMENT ---
%
\begin{document}
\setsecnumdepth{subsubsection}
\maketitle
%\setsecnumdepth{subsection}
\thispagestyle{courseinformationtitle}
%
% --- INSTRUCTOR ---
%
\section{Instructor}
\begin{tabular}{rl|rl}
Name: & Dr. Dale J. Brugh & Office Phone: & 740-368-3530 \\
Email: & \href{mailto:djbrugh@owu.edu}{djbrugh@owu.edu} & Cell Phone: & 614-746-2397 \\
Office: & SCSC 262 & &  \\
\end{tabular}
%
% --- MEETINGS ---
%
\section{Meetings}
\begin{tabular}{crcrl}
TR & 8:00 a.m. & to & 9:50 a.m. & SCSC 167\\
\end{tabular} \\[0.09in]
Optional review meetings are scheduled on Wednesday evenings from 7:30 p.m.\ to 9:00 p.m. These meetings are to provide help with concepts and homework. 
%
% --- WEBSITE ---
%
\section{Website}
The course website is located at \href{http://dephlo.org/pchem}{dephlo.org/pchem}. The website is an important extension of the syllabus and should be read carefully.
%
% --- PREREQUISITES ---
%
\section{Prerequisites}
To take this course you must have passed two semesters of general chemistry, two semesters of organic chemistry, two semesters of physics, two semesters of calculus, and one semester of physical chemistry that covers quantum mechanics.
%
% --- MATERIALS ---
%
\section{Materials}
The following items are required for this course. Additional details can be found on the course website at \href{http://dephlo.org/pchem-materials}{dephlo.org/pchem-materials}.
\begin{enumerate}
\item \emph{Physical Chemistry} with MasteringChemistry by Thomas Engel and Phillip Reid, Third Edition, Prentice Hall (2013).  ISBN-13: 9780321766205
\item \emph{Quanta, Matter, and Change} by Peter Atkins, Julio de Paula, and Ronald Friedman, First Edition, W.H. Freeman and Company (2009). ISBN-10: 07167-6117-3, ISBN-13: 978-0-7167-6117-4.
\item Wolfram's \emph{Mathematica}.
\item A working scientific calculator.
\end{enumerate}
%
% --- CONTENT ---
%
\section{Content}
This course is an introduction to statistical mechanics, thermodynamics, and chemical kinetics with an emphasis on molecular derivation and interpretation of macroscopic properties and behaviors. The course topics are listed in Table~\ref{tab:topics}.  
%
% --- Table: Course Topics ---
%
\begin{table}[h]
\caption{\sffamily Topics covered in Chemistry 351}
\label{tab:topics}
\renewcommand{\arraystretch}{1}
\begin{tabular}{l|l|l} \toprule
Probability and Statistics & Second Law & Real Gases \\
Boltzmann Distribution  & Physical Equilibrium & Chemical Kinetics \\
Partition Functions      & Chemical Equilibrium & Reaction Mechanisms \\
Statistical Thermodynamics    & Thermochemistry & Chemical Dynamics \\
First Law     & Kinetic Gas Theory & Catalysis \\
\bottomrule
\end{tabular}
\end{table}
\marginnote{-0.55in}{Topics are not necessarily covered in this order, and not all topics are covered in equal depth.}

Statistical mechanics connects the properties of atoms and molecules to the bulk properties of matter manifested in classical thermodynamics. Classical thermodynamics allows the behavior of bulk matter to be predicted without any knowledge of its detailed structure. Chemical kinetics and dynamics allow reaction mechanisms to be predicted and the progress of reactions to be modeled. With these tools, you can predict how Nature behaves under a broad range of conditions. \marginnote{-0.35in}{These are basic tools in any chemist's toolbox for working on chemical problems.}
%
% --- GOALS ---
%
\section{Goals}
In this class I want to
\begin{inparaenum}[\bfseries (a\upshape)]
\item provide you with the tools to understand how statistical mechanics, thermodynamics, and kinetics apply to chemical systems;
\item provide you with the tools to understand statistical mechanics, thermodynamics, and kinetics in the literature of any chemical subject;
\item show you the beauty and regularity of chemistry and Nature that emerge when you understand the most fundamental concepts that underlie everything you have ever experienced;
\item provide you with the tools to form a realistic mental image of how microscopic properties give rise to macroscopic properties;
\item further develop your ability to analyze and interpret experimental results in terms of the microscopic structure and dynamics of atoms and molecules, and 
\item help you understand how we know what we know.
\end{inparaenum}

I also want you to move closer to being a professional in this course. Because of this, additional goals of this course are to
\begin{inparaenum}[\bfseries (a\upshape)]
\item improve your problem solving skills;
\item improve your ability to formulate descriptions of the chemical world in terms of mathematical models, and
\item improve your ability to make clear quantitative arguments.
\end{inparaenum}
%
% --- LEARNING OBJECTIVES ---
%
\section{Learning Objectives}
Learning objectives are things you should be able to do at the end of the course. For each topic covered in this course, you are provided with a list of learning objectives called Be Able Tos, or BATS for short. A complete list of BATS for each topic can be found on the web page page for that topic. These BATS are detailed (granular), and it might be difficult to see the overall objectives of the course from them.

 A higher level (less granular) list of learning objectives might be helpful. At the end of the course, you should be able to
\begin{inparaenum}[\bfseries (a\upshape)]
\item explain the meaning and origin of Boltzmann's distribution;
\item calculate thermodynamic properties using molecular properties;
\item state and apply the laws of thermodynamics;
\item use thermodynamic engines to discuss efficiency;
\item predict chemical equilibrium and spontaneity of reactions;
\item use equations of state to determine properties of gaseous systems;
\item explain gas behavior using kinetic gas theory;
\item perform calculations to determine reaction rates;
\item analyze reaction mechanisms in terms of their rate laws, and 
\item model concentration as a function of time for the species involved in any mechanism.
\end{inparaenum}
%
% --- TIME REQUIREMENT ---
%
\section{Time Requirement}
Each \marginnote{-0.1in}{The time required will depend on your level of preparation and ability to focus. Estimates are for a typical student.} of the two weekly course meetings is 110 minutes in length, requiring a total of 3.67 hours per week. For each meeting you can expect to spend about 2 hours outside of class reading, studying, and working exercises. Problem sets will require about five to ten hours during weeks they are assigned. This course will typically require time over weekends. 
%
% --- WEEKLY ROUTINE ---
%
\section{Weekly Routine}
Tuesday and Thursday meetings are primarily dedicated to lecture. An exercise assignment is due at the start of most lecture meetings. A problem set is due most Thursdays. An optional review meeting is offered Wednesday evening. There are three micro exams and three exams. All are given on Thursdays during a normal class meeting. A detailed course schedule can be found on the course website. 
%
% --- THINGS I GRADE ---
%
\section{Things I Grade}
You \marginnote{-0.1in}{The course website lists due dates for all evaluated items.} and I determine your progress in this course using scores derived from evaluating the quality and accuracy of your answers to questions posed in exercise assignments, problem sets, micro exams, exams, and a final exam. This section provides details for each of these evaluations
%
% --- Exercise Assignments ---
%
\subsection{Exercise Assignments}
Exercise \marginnote{-0.1in}{Worth 15\% of course score. Lowest 6 scores are dropped.} assignments are assigned after most TR lecture meetings and consist of three questions. You can download them from the course website. For each exercise assignment you are to write out solutions on paper. These hand-written solutions are due at the start of the lecture meeting indicated in the course schedule. Solutions will be available on the course website after the assignment is graded. Exercise assignments are worth 15 points each. Your lowest six (6) exercise assignment scores are dropped before calculating your course score. You are encouraged to work with other students while completing exercise assignments.
%
% --- Problem Sets ---
%
\subsection{Problem Sets}
Problem \marginnote{-0.1in}{Worth 15\% of course score. Lowest one (1) score is dropped.} sets are more challenging than exercise assignments; it is best to think of them as projects. They are typically assigned weekly except during weeks with exams; they are due on Thursdays. Problem sets are posted on the course website, and solutions are available in my office. Each is worth 100 points. Your lowest single problem set score is dropped before computing your course score. The course website has detailed guidelines for preparing solutions for problems sets.
%
% --- Micro Exams ---
%
\subsection{Micro Exams}
A \marginnote{-0.1in}{Worth 15\% of course score. Lowest one (1) score is dropped.} micro exam is given about two weeks prior to each exam at the start of a Thursday meeting. Exact dates are in the course schedule. You will be given 50 minutes to complete each micro exam. The topics covered on each micro exam are listed on the course website. Solutions to micro exams are posted on the course website after they are graded. Each micro exam is worth 50 points. Your lowest single micro exam score is dropped before computing your course score. 
%
% --- Exams ---
%
\subsection{Exams}
There \marginnote{-0.1in}{Worth 30\% of course score.} are three exams during the semester. Each is given during a normal Thursday class meeting. They require the entire class period to complete. All exams are cumulative. Solutions for exams are available in my office. See the course website for the exam schedule. Each exam is weighted equally. 
%
% --- Final Exam ---
%
\subsection{Final Exam}
The \marginnote{-0.1in}{Worth 25\% of course score.} final exam is given on the day and at the time specified by the Registrar. The final exam is three hours in length, and it is cumulative over the entire semester. No solutions to the final exam are posted, but you may review the grading by making an appointment with me. You may not take possession of the exam. 
%
% --- THINGS I DO NOT GRADE ---
%
\section{Things I Do Not Grade}
Each topic is accompanied by a set of practice exercises from the required textbooks. These exercises are recommended, but I never collect and grade them. It is, however, essential that you work (or at least review) as many of the practice exercises as possible. Solutions for practice exercises are available outside my office. 
%
% --- COURSE SCORE ---
%
\section{Course Score}
Your course score is a weighted average of the scores you earn on exercise assignments, problem sets, micro exams, exams, and the final exam. These scores are weighted according to the percentages shown in Table~\ref{tab:weights} on page~\pageref{tab:weights}. Before calculating your course score, I will throw away your lowest six (6) exercise assignment scores, your lowest one (1) problem set score, and your lowest one (1) micro exam score. 
%
% --- Table: Evaluation Weights
%
\renewcommand{\arraystretch}{1.15}
\begin{table}[h]
\caption{\sffamily Weight of evaluated items}
\label{tab:weights}
\begin{tabular}{rc|rc}
\hline\hline
\textbf{Evaluation Item} & \textbf{Weight} & \textbf{Evaluation Item} & \textbf{Weight} \\ \hline
Exercise Assignments & 15\% & Exams & 30\% \\
Problem Sets & 15\% & Final Exam & 25\% \\
Micro Exams & 15\% &  \\ 
\hline\hline
\end{tabular}
\end{table}
%
% --- LETTER GRADE ---
%
\section{Letter Grade}
Letter grades are assigned at the end of the course according to the minimum course score requirements listed in Table~\ref{tab:lettergrades}. Course scores below $55\%$ are considered failing. Please see \href{http://dephlo.net/lettergrades}{dephlo.net/lettergrades} for more detail about how your course letter grade is determined. 
\begin{table}[h]
\caption{\sffamily Minimum course scores necessary for each letter grade.}
\label{tab:lettergrades}
\begin{tabular}{cl||cl} \toprule
\textbf{Minimum Score} & \textbf{Letter Grade} & \textbf{Minimum Score} & \textbf{Letter Grade} \\ \hline
97 & \hspace{0.3in}A$+$ & 72 & \hspace{0.3in}C$+$ \\
88 & \hspace{0.3in}A & 68 & \hspace{0.3in}C \\
85 & \hspace{0.3in}A$-$ & 65 & \hspace{0.3in}C$-$ \\
82 & \hspace{0.3in}B$+$ & 62 & \hspace{0.3in}D$+$ \\
78 & \hspace{0.3in}B & 58 & \hspace{0.3in}D \\
75 & \hspace{0.3in}B$-$ & 55 & \hspace{0.3in}D$-$ \\
\bottomrule
\end{tabular}
\end{table}
%
% --- ADDITIONAL INFORMATION ---
%
\section{Additional Information}
Please see the course website at \href{http://dephlo.net/pchem}{dephlo.net/pchem} for additional information such as suggestions for success, detailed course policies, problem set guidelines, course schedule, and solutions. 
%
% ---LICENSE AND SOURCE CODE ---
%
\section{License and Source Code}
\copyright\ 2014 by Dale J. Brugh. Course Syllabus for Chemistry 351 (2015) is made available under a Creative Commons Attribution-ShareAlike 4.0 International License (CC BY-SA 4.0). See \href{https://creativecommons.org/licenses/by-sa/4.0/}{https://creativecommons.org/licenses/by-sa/4.0/} for details about your rights under this license. The \LaTeX\ source code is available under the same license from Github at \href{https://github.com/djbrugh/chem351}{https://github.com/djbrugh/chem351}. \marginpar{\vspace{-0.85in}\hfill\noindent\href{https://creativecommons.org/licenses/by-sa/4.0/}{\includegraphics[width=1in]{figs/cc-by-sa.pdf}}}

\end{document}